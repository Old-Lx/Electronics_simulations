\documentclass[12pt]{article}
\usepackage[utf8]{inputenc}
\usepackage[T1]{fontenc}
\usepackage{geometry}
\usepackage{graphicx}
\usepackage{makeidx}
\geometry{margin=2.5cm}
\usepackage{fancyhdr}

\begin{document}
	
    \thispagestyle{empty}
	
    \begin{center}
        \includegraphics[width=3.1cm,height=2cm]{logo}\\
        UNIVERSIDAD SIMÓN BOLÍVAR\\
        DEPARTAMENTO DE ELECTRÓNICA Y CIRCUITOS\\
        EC1281 - LABORATORIO DE MEDICIONES ELÉCTRICAS\\
        SECCIÓN 1 - GRUPO 1\\
        
        \vspace{7cm}
        \textbf{\Large INFORME - PRÁCTICA \#6}\\
        MEDICIONES EN AC CON EL OSCILOSCOPIO CIRCUITO RLC SERIE\\
    \end{center}
    
    \begin{flushleft}
        \vspace{9cm}
        \hfill Integrantes:\\
        \hfill {\large Luis Becerra - 1910557}\\
        \hfill {\large Lorena Rojas - 1910469}\\
    \end{flushleft}
    
    \newpage
    
    \pagenumbering{Roman}
        \setcounter{page}{2}
    
    \begin{center}
        \textbf{\large RESUMEN}\\
    \end{center}
	
    En este informe de laboratorio de mediciones eléctricas, el objetivo fue utilizar el osciloscopio para observar formas de onda, medir amplitudes, frecuencias y desfasajes de señales eléctricas, así como estudiar las respuestas al escalón de circuitos RLC serie y analizar la respuesta en frecuencia de un circuito RLC serie utilizado como filtro pasabanda, elimina-banda, pasa-bajo y pasa-alto. El procedimiento consistió en montar y ajustar los circuitos, realizar mediciones en el osciloscopio y registrar los valores correspondientes. Los resultados más relevantes incluyeron la identificación de los tipos de respuesta de los circuitos RLC, la determinación de la frecuencia de resonancia, frecuencias de corte, ancho de banda y factor de calidad. Las conclusiones fundamentales indicaron que los circuitos RLC presentan diferentes comportamientos dependiendo de las condiciones de amortiguamiento, y su respuesta en frecuencia permite su utilización como filtros en distintas configuraciones. 
    
    \newpage
    
    \begin{center}
        \textbf{\large ÍNDICE}\\
    \end{center}
    
    \noindent \textbf{RESUMEN} \hfill \textbf{II}\\
    \noindent \textbf{ÍNDICE} \hfill \textbf{III}\\
    \noindent \textbf{MARCO TEÓRICO} \hfill \textbf{1}\\
    \noindent \textbf{METEDOLOGÍA} \hfill \textbf{3}\\
    \noindent \textbf{RESULTADOS} \hfill \textbf{5}\\
    \noindent \textbf{ANÁLISIS DE RESULTADOS} \hfill \textbf{}\\
    \noindent \textbf{CONCLUSIONES} \hfill \textbf{}\\
    \noindent \textbf{BIBLIOGRAFÍA} \hfill \textbf{}\\
    \noindent \textbf{ANEXOS} \hfill \textbf{}\\
    
    \newpage
    
    \pagenumbering{arabic}
    
    \begin{center}
        \textbf{\large MARCO TEÓRICO}\\
    \end{center}
    
    \textbf{1. Circuitos RLC serie}\\

    \textbf{1. 1. Parámetros del circuito RLC serie:}

    \begin{itemize}
        \item \textbf{Frecuencia de resonancia ($f_{r}$):} La frecuencia de resonancia en un circuito RLC serie es aquella en la cual la impedancia total del circuito alcanza su valor mínimo. En esta frecuencia, la reactancia inductiva y la reactancia capacitiva se cancelan mutuamente, resultando en una respuesta de frecuencia máxima. La frecuencia de resonancia se calcula utilizando la fórmula $f_{r} = 1 / (2\pi\sqrt{LC})$, donde L es la inductancia y C es la capacitancia del circuito.
    
        \item \textbf{Frecuencia de corte inferior ($f_{1}$):} La frecuencia de corte inferior es aquella a la cual la respuesta en frecuencia del circuito RLC serie comienza a disminuir por debajo de la respuesta en frecuencia máxima. En esta frecuencia, la reactancia inductiva y la resistencia determinan la respuesta del circuito. La frecuencia de corte inferior se puede calcular utilizando la fórmula $f_{1} = 1 / (2\pi\sqrt{LC}) - (R / (2L))$.
    
        \item \textbf{Frecuencia de corte superior ($f_{2}$):} La frecuencia de corte superior es aquella a la cual la respuesta en frecuencia del circuito RLC serie comienza a disminuir por encima de la respuesta en frecuencia máxima. En esta frecuencia, la resistencia y la reactancia capacitiva determinan la respuesta del circuito. La frecuencia de corte superior se puede calcular utilizando la fórmula $f_{2} = 1 / (2\pi\sqrt{LC}) + (R / (2L))$.
    
        \item \textbf{Ancho de banda (BW):} El ancho de banda en un circuito RLC serie es la diferencia entre la frecuencia de corte superior y la frecuencia de corte inferior. Representa el rango de frecuencias en el cual la respuesta en frecuencia del circuito se mantiene dentro de un margen aceptable. El ancho de banda se puede calcular como $BW = f_{2} - f_{1}$.
    
        \item \textbf{Factor de calidad (Q):} El factor de calidad en un circuito RLC serie es una medida de la selectividad del circuito. Se calcula dividiendo la frecuencia de resonancia ($f_{r}$) por el ancho de banda (BW). Un valor alto de Q indica un circuito más selectivo, con una respuesta en frecuencia más estrecha. El factor de calidad se puede calcular como $Q = f_{r} / BW$.
    \end{itemize}
    
    \textbf{1. 2. Configuraciones de filtro con el circuito RLC serie:}

    \begin{itemize}
        \item \textbf{Filtro pasa-bajo pasivo:} Un filtro pasa-bajo pasivo se puede obtener utilizando la salida del voltaje sobre la resistencia (R). En esta configuración, el circuito permite el paso de las frecuencias más bajas mientras atenúa gradualmente las frecuencias más altas. Es útil para filtrar señales de alta frecuencia y mantener las componentes de baja frecuencia.
    
        \item \textbf{Filtro pasa-alto pasivo:} Un filtro pasa-alto pasivo se obtiene considerando la salida del voltaje sobre el inductor (L). En esta configuración, el circuito permite el paso de las frecuencias más altas mientras atenúa las frecuencias más bajas. Es utilizado para eliminar las componentes de baja frecuencia y resaltar las señales de alta frecuencia.
    
        \item \textbf{Filtro pasa-banda pasivo:} Para obtener un filtro pasa-banda pasivo, se considera el voltaje sobre la resistencia (R) como salida. Este tipo de filtro permite el paso de un rango específico de frecuencias, atenuando las frecuencias fuera de ese rango. Es utilizado para seleccionar y amplificar una banda estrecha de frecuencias en una señal.
    
        \item \textbf{Filtro elimina-banda pasivo:} Un filtro elimina-banda pasivo se obtiene considerando el voltaje sobre la conexión en serie del condensador (C) y el inductor (L). En esta configuración, el circuito atenúa un rango específico de frecuencias, permitiendo el paso de las frecuencias fuera de ese rango. Es utilizado para eliminar una banda específica de frecuencias en una señal.
    \end{itemize}
    
    Estas configuraciones del circuito RLC serie permiten implementar diferentes tipos de filtros pasivos para el procesamiento de señales en diversas aplicaciones.
    
    \newpage
    
    \begin{center}
        \textbf{\large METODOLOGÍA}\\
    \end{center}
    
    \textbf{1. Procedimiento para determinar experimentalmente la frecuencia de resonancia ($f_{r}$), las frecuencias de corte superior ($f_{2}$) e inferior ($f_{1}$), el ancho de banda ($BW = f_{2}-f_{1}$) y el factor de calidad (Q) del circuito RLC serie:}

    \begin{itemize}
        \item Se monta el circuito RLC serie según la configuración especificada en el diseño.
        \item Se ajusta el generador de funciones para obtener una señal sinusoidal de amplitud constante y frecuencia variable.
        \item Se aplica la señal sinusoidal al circuito y se observa la respuesta en el osciloscopio.
        \item Se varía la frecuencia del generador y se registran los valores de voltaje sobre la resistencia en el circuito.
        \item Se identifica la frecuencia de resonancia ($f_{r}$) donde el voltaje sobre la resistencia es máximo y la fase es cero, esto se logra variando la frecuencia del generador de funciones.
        \item Se determinan las frecuencias de corte superior ($f_{2}$) e inferior ($f_{1}$), variando la frecuencia del generador de funciones, donde el voltaje sobre la resistencia es el 70.7\% de su valor máximo.
        \item Se calcula el ancho de banda (BW) restando $f_{1}$ de $f_{2}$.
        \item Se calcula el factor de calidad (Q) dividiendo la frecuencia de resonancia ($f_{r}$) entre el ancho de banda (BW).
    \end{itemize}
    
    \textbf{2. Mediciones sobre cada uno de los filtros en un circuito RLC en serie:}

    \begin{itemize}
        \item \textbf{Filtro pasa-bajo pasivo:} Se medirá la amplitud y el desfase de la señal de salida (voltaje sobre la resistencia) en función de la frecuencia de entrada.
        \item \textbf{Filtro pasa-alto pasivo:} Se medirá la amplitud y el desfase de la señal de salida (voltaje sobre el inductor) en función de la frecuencia de entrada.
        \item \textbf{Filtro pasa-banda pasivo:} Se medirá la amplitud y el desfase de la señal de salida (voltaje sobre la resistencia) en función de la frecuencia de entrada.
        \item \textbf{Filtro elimina-banda pasivo:} Se medirá la amplitud y el desfase de la señal de salida (voltaje sobre el condensador) en función de la frecuencia de entrada.
    \end{itemize}
    
    \textbf{3. Configuración para obtener la respuesta transitoria al escalón sobre el condensador de un circuito RLC serie:}\\
    
    Consiste en montar el circuito RLC serie con el generador de funciones conectado al circuito y un osciloscopio para observar la respuesta. Las mediciones que se van a realizar son:

    \begin{itemize}
        \item Se aplica un escalón de voltaje al circuito y se registra la respuesta en el osciloscopio.
        \item Se mide el tiempo de subida, el tiempo de establecimiento y el tiempo de caída de la respuesta transitoria.
        \item Se determina el sobrepico máximo y se calcula el factor de amortiguamiento.
        \item Se realiza un análisis cualitativo de la respuesta transitoria, observando el comportamiento del circuito en términos de oscilaciones, estabilidad y decaimiento de la señal.
        \item Estas mediciones permitirán caracterizar la respuesta transitoria del circuito RLC serie y evaluar su comportamiento en relación con el escalón aplicado.
    \end{itemize}
    
    \newpage
    
    \begin{center}
        \textbf{\large RESULTADOS}\\
    \end{center}
    
    Inserte resultados
    
    \newpage
    
    \begin{center}
        \textbf{\large ANÁLISIS DE RESULTADOS}\\
    \end{center}
    
    Inserte análisis de resultados
    
    \newpage
    
    \begin{center}
        \textbf{\large CONCLUSIONES}\\
    \end{center}
    
    Inserte conclusiones
    
    \newpage
    
    \begin{center}
        \textbf{\large BIBLIOGRAFÍA}\\
    \end{center}
    
    \noindent Laboratorios de Circuitos Electrónicos, Guía Teórica versión electrónica, ubicada en la página web del laboratorio C, http://www.labc.usb.ve, enlace a "Páginas web de Asignaturas", EC1281- Laboratorio de Mediciones Eléctricas 2016.
    
    \newpage
    
    \begin{center}
        \textbf{\large ANEXOS}\\
    \end{center}
    
    Inserte anexos
	
\end{document}