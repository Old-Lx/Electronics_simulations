\documentclass[12pt]{article}
\usepackage[utf8]{inputenc}
\usepackage[T1]{fontenc}
\usepackage{geometry}
\usepackage{graphicx}
\usepackage{makeidx}
\geometry{margin=2.5cm}
\usepackage{fancyhdr}

\begin{document}
	
	\thispagestyle{empty}
	
	\begin{center}
		\includegraphics[width=3.1cm,height=2cm]{logo}\\
		UNIVERSIDAD SIMÓN BOLÍVAR\\
		DEPARTAMENTO DE ELECTRÓNICA Y CIRCUITOS\\
		EC1281 - LABORATORIO DE MEDICIONES ELÉCTRICAS\\
		SECCIÓN 1 - GRUPO 1\\
		
		\vspace{7cm}
		\textbf{\Large INFORME - PRÁCTICA \#7}\\
		INSTRUMENTO DE MEDICÓN PARA CORRIENTE ALTERNA (AC)\\
	\end{center}
	
	\begin{flushleft}
		\vspace{9cm}
		\hfill Integrantes:\\
		\hfill {\large Luis Becerra - 1910557}\\
		\hfill {\large Lorena Rojas - 1910469}\\
	\end{flushleft}
	
	\newpage
	
	\pagenumbering{Roman}
        \setcounter{page}{2}
	
	\begin{center}
		\textbf{\large RESUMEN}\\
	\end{center}
	
	En esta práctica de laboratorio, se tuvieron como objetivos principales la interpretación de las características nominales de los instrumentos de medición para corriente alterna (AC) y el uso adecuado de dichos instrumentos en mediciones en AC. Se realizaron mediciones directas e indirectas de voltajes y corrientes pico y rms, mediciones de frecuencia, desfasaje y potencia en circuitos alimentados con fuentes alternas. Además, se compararon los resultados experimentales con los valores teóricos o simulados de los circuitos. Las mediciones se llevaron a cabo utilizando voltímetros, amperímetros, multímetros y osciloscopios. Se realizaron mediciones de resistencias, voltajes pico y rms, corrientes pico, desfasaje y potencia en diferentes configuraciones de circuitos. Los resultados obtenidos fueron analizados y se concluyó sobre la precisión y confiabilidad de los instrumentos de medición utilizados.
	
	\newpage
	
	\begin{center}
		\textbf{\large ÍNDICE}\\
	\end{center}
	
	\noindent \textbf{RESUMEN} \hfill \textbf{II}\\
	\noindent \textbf{ÍNDICE} \hfill \textbf{III}\\
	\noindent \textbf{MARCO TEÓRICO} \hfill \textbf{1}\\
	\noindent \textbf{METEDOLOGÍA} \hfill \textbf{}\\
	\noindent \textbf{RESULTADOS} \hfill \textbf{}\\
	\noindent \textbf{ANÁLISIS DE RESULTADOS} \hfill \textbf{}\\
	\noindent \textbf{CONCLUSIONES} \hfill \textbf{}\\
	\noindent \textbf{BIBLIOGRAFÍA} \hfill \textbf{}\\
	\noindent \textbf{ANEXOS} \hfill \textbf{}\\
	
	\newpage
	
	\pagenumbering{arabic}
	
	\begin{center}
		\textbf{\large MARCO TEÓRICO}\\
	\end{center}
	
	\textbf{1. Valor medio cuadrático (RMS) de una señal alterna}\\
	
	El valor medio cuadrático (RMS, por sus siglas en inglés) es una medida importante para caracterizar una señal alterna. Representa el valor efectivo de la amplitud de una señal y se calcula mediante la raíz cuadrada de la media del cuadrado de los valores instantáneos de la señal a lo largo de un período.\\
	
	Para una señal periódica $V(t)$ con un período $T$, el valor RMS se define como: $$V(t) = \sqrt{\frac{1}{T}*\int_{0}^{T}V^2(t)dt}$$	
	
	Donde $V(t)$ es el valor instantáneo de la señal en un instante de tiempo $t$. El valor RMS es útil porque representa la amplitud equivalente de una señal de corriente alterna en términos de su valor continuo.\\
	
	\textbf{2. Ancho de banda de un circuito o instrumento de corriente alterna}\\
	
	El ancho de banda de un circuito o instrumento de corriente alterna se refiere al rango de frecuencias dentro del cual la respuesta del sistema es aceptablemente buena. Se define como la diferencia entre la frecuencia más alta y la frecuencia más baja que puede ser transmitida o medida por el sistema sin una degradación significativa.\\
	
	En general, el ancho de banda está relacionado con la capacidad del circuito o instrumento para responder a señales de diferentes frecuencias. Los circuitos o instrumentos con un ancho de banda amplio pueden capturar o medir señales de frecuencia más alta, mientras que aquellos con un ancho de banda más estrecho tienen una respuesta limitada a frecuencias altas.\\
	
	El ancho de banda puede estar determinado por la capacidad de los componentes del circuito, como condensadores e inductores, así como por la respuesta de los amplificadores o filtros utilizados en el sistema. Es importante considerar el ancho de banda al seleccionar un circuito o instrumento de corriente alterna para asegurarse de que pueda manejar las frecuencias de interés.\\
	
	\textbf{3. Instrumentos de medición de voltajes y corrientes AC}\\
	
	Existen varios tipos de instrumentos de medición utilizados para medir voltajes y corrientes en circuitos de corriente alterna (AC). Estos instrumentos ofrecen diferentes características y se utilizan en diferentes aplicaciones según las necesidades del usuario.
	
	\begin{itemize}
		\item \textbf{Voltímetro de AC:} Este instrumento se utiliza para medir el valor eficaz o RMS de un voltaje AC. Puede tener escalas y rangos variables para adaptarse a diferentes niveles de voltaje. Algunos voltímetros de CA también pueden medir frecuencia y desfasaje.
		
		\item \textbf{Amperímetro de AC:} Este instrumento se utiliza para medir el valor eficaz o RMS de una corriente AC. Al igual que el voltímetro de CA, puede tener escalas y rangos variables para adaptarse a diferentes niveles de corriente. Algunos amperímetros de CA también pueden medir frecuencia y desfasaje.
		
		\item \textbf{Multímetro:} Es un instrumento versátil que combina funciones de voltímetro, amperímetro y ohmímetro. Puede medir voltajes y corrientes AC en diferentes rangos, así como resistencias y otras magnitudes eléctricas. Algunos multímetros también tienen capacidades de medición de frecuencia y desfasaje.
		
		\item \textbf{Osciloscopio:} Es un instrumento utilizado para visualizar y analizar formas de onda de señales eléctricas. Puede mostrar la amplitud, frecuencia, forma de onda y desfasaje de una señal AC en una pantalla. Los osciloscopios también pueden realizar mediciones precisas de voltajes y tiempos en señales AC.
		
	\end{itemize}
	
	Cada tipo de instrumento tiene sus ventajas y limitaciones, y la elección depende del tipo de medición requerida, el rango de frecuencias de interés y la precisión necesaria.

	\newpage
	
	\begin{center}
		\textbf{\large METODOLOGÍA}\\
	\end{center}
	
	Inserte metodología
	
	\newpage
	
	\begin{center}
		\textbf{\large RESULTADOS}\\
	\end{center}
	
	A continuación se presentan los resultados obtenidos durante el trabajo en el laboratorio:
	
	\includegraphics[width=16cm,height=21cm]{Img/Resultados_1}\\
	\includegraphics[width=16cm,height=21cm]{Img/Resultados_2}\\
	\includegraphics[width=16cm,height=21cm]{Img/Resultados_3}\\
	\includegraphics[width=16cm,height=21cm]{Img/Resultados_4}\\
	\includegraphics[width=16cm,height=21cm]{Img/Resultados_5}\\
	\includegraphics[width=16cm,height=21cm]{Img/Resultados_6}\\
	
	\noindent 15.- Ubicadas las puntas de prueba según lo indicado, realice la medición del voltaje pico en la rama $Z_{C}$, el voltaje pico sobre la resistencia $R_{2}$ para calcular la corriente pico en dicha rama y el desfasaje entre ambas señales, el cual es el ángulo de la impedancia $Z_{C}$. Calcule el módulo de dicha impedancia ($V/I$) y registre todos los valores en las casillas correspondientes. Tome una foto de las señales observadas.
	
	\begin{center}
		\includegraphics[width=10cm,height=6cm]{Img/resul_15}\\
	\end{center}
	
	\noindent 16.- Para determinar experimentalmente la impedancia de la rama inductiva ZL coloque ahora la punta del CH1 en el nodo B y la del CH2 en el nodo E, con la tierra en el nodo F, como se indica en la Figura 8. Realice las mediciones del voltaje pico en la rama, el voltaje pico sobre la resistencia R3 para calcular la corriente pico por la rama, y el desfasaje entre ambas señales. Calcule el módulo de la impedancia para este caso, registre los datos en las casillas disponibles y tome una foto de las señales observadas.
	
	\begin{center}
		\includegraphics[width=10cm,height=5cm]{Img/Captura}\\
		\includegraphics[width=10cm,height=6cm]{Img/resul_16}\\
	\end{center}
	
	\includegraphics[width=16cm,height=21cm]{Img/Resultados_7}\\
	\includegraphics[width=16cm,height=21cm]{Img/Resultados_8}\\
	\includegraphics[width=16cm,height=21cm]{Img/Resultados_9}\\
	
	\begin{center}
		\includegraphics[width=12cm,height=8cm]{Img/resul_22_1}\\
		\textit{Potencia Máxima}\\
		\vspace{2cm}
		\includegraphics[width=12cm,height=8cm]{Img/resul_22_2}\\
		\textit{A la mitad del voltaje pico}
	\end{center}

	\includegraphics[width=16cm,height=21cm]{Img/Resultados_10}\\
	
	\begin{center}
		\includegraphics[width=12cm,height=8cm]{Img/resul_24_1}\\
		\textit{Potencia Máxima}\\
		\vspace{2cm}
		\includegraphics[width=12cm,height=8cm]{Img/resul_24_2}\\
		\textit{A la mitad de la forma de la onda sinusoidal}\\
	\end{center}
	
	\newpage
	
	\begin{center}
		\textbf{\large ANÁLISIS DE RESULTADOS}\\
	\end{center}
	
	Inserte análisis de resultados
	
	\newpage
	
	\begin{center}
		\textbf{\large CONCLUSIONES}\\
	\end{center}
	
	Inserte conclusiones
	
	\newpage
	
	\begin{center}
		\textbf{\large BIBLIOGRAFÍA}\\
	\end{center}
	
	Inserte bibliografía
	
	\newpage
	
	\begin{center}
		\textbf{\large ANEXOS}\\
	\end{center}
	
	Inserte anexos
	
\end{document}
